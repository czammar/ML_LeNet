\documentclass[xcolor=dvipsnames, compress]{beamer}
%\usetheme{Madrid} % My favorite!
%\usetheme{Boadilla} % Pretty neat, soft color.
%\usetheme{Warsaw}
%\usecolortheme{dove} 
%\usetheme[secheader]{Boadilla}
%\useoutertheme[subsection=false]{smoothbars}
% \useinnertheme{rectangles}
% \usetheme{Marburg}
%   \usecolortheme[RGB={139,10,80}]{structure}
%  \usecolortheme[RGB={25,25,112}]{structure}
%\usecolortheme[RGB={255,127,36}]{structure}
%\usetheme{CambridgeUS}
%\usetheme{PaloAlto}
% \usefonttheme{professionalfonts}
% \usepackage{listings}

%\usetheme{Boadilla}

% \usetheme{Warsaw}
%\usetheme{Darmstadt} %OK!
%  \usetheme{Frankfurt} %OK!
% \usetheme{Goettingen}
% \usetheme{Dresden}
%\usetheme{JuanLesPins} %OK!!
%\usetheme{Marburg}
%  \usetheme{Montpellier}
% \usetheme{Rochester} %sobrio
%\usetheme{Singapore}
%\usetheme{Szeged}
%\usetheme{Luebeck}

%\usetheme{Hannover}
%\usecolortheme{wolverine}

%\usecolortheme{albatross}
%\usecolortheme{seahorse}
%\usecolortheme{beetle}
%\usecolortheme{crane}
%\usecolortheme{dolphin}
%\usecolortheme{dove} %<- este con orchid
%\usecolortheme{fly}
%\usecolortheme{lily}
%\usecolortheme{orchid}
%\usecolortheme{rose}
%\usecolortheme{seagull}
%\usecolortheme{whale}

%\usetheme{Bergen} % This template has nagivation on the left
%\usetheme{Frankfurt} % Similar to the default 
%with an extra region at the top.
\usecolortheme{seahorse} % Simple and clean template
%\usetheme{Darmstadt} % not so good
% Uncomment the following line if you want %
% page numbers and using Warsaw theme%
% \setbeamertemplate{footline}[page number]
%\setbeamercovered{transparent}
%\setbeamercovered{invisible}
% To remove the navigation symbols from 
% the bottom of slides%
%\setbeamertemplate{navigation symbols}{} 
%
\usepackage{graphicx}
\usepackage{amssymb,amsmath,amscd}
\usepackage{latexsym,xspace}
\usepackage[utf8]{inputenc}
\usepackage{epsfig}
%\usepackage{fancyhdr}
%\usepackage[spanish]{babel}
\usepackage[all]{xy}
\usepackage{enumerate}
\usepackage{eucal}
%\usepackage[usenames]{color}

\usepackage{mathtools} % flechas con nombres arriba o abajo


%#########################
\newcommand{\tx}{\ensuremath{\tau(X)}}
\newcommand{\txx}{\ensuremath{\tau_{X}}}
\newcommand{\Q}{\ensuremath{\mathbb{Q}}}
\newcommand{\Z}{\ensuremath{\mathbb{Z}}}
\newcommand{\N}{\ensuremath{\mathbb{N}}}
\newcommand{\R}{\ensuremath{\mathbb{R}}}
\newcommand{\C}{\ensuremath{\mathbb{C}}}
\newcommand{\A}{\ensuremath{\forall}}
\newcommand{\E}{\ensuremath{\exists}}
\newcommand{\iso}{\ensuremath{\cong}}
\newcommand{\union}{\ensuremath{\cup}}
%\newcommand{\morinyec}{\ensuremath{\precapprox}}
%\newenvironment{prueba}{\vspace{-3mm}\noindent\textbf{Demostraci\'on}\\}{\noindent$\blacksquare$\\}
\newcommand{\nin}{\ensuremath{\notin}}
\renewcommand{\emptyset}{\varnothing}

%\newcommand{\niso}{\ensuremath{\not \cong}}
\newtheorem{teor}{Teorema}[section]
\newtheorem{defi}{Definition}[section]
\newtheorem{ejemplo}{Examples}[section]
\newtheorem{obs}{Remark}[section]
\newtheorem{prop}{Proposition}[section]
\newtheorem{cor}{Corollary}[section]
\newtheorem{ntc}{Notation}[section]
\newtheorem{lema}{Lemma}[section]
\newtheorem{prob}{Problem}
\newtheorem{comen}{Comment}


%\usepackage{bm}         % For typesetting bold math (not \mathbold)
%\logo{\includegraphics[height=0.6cm]{yourlogo.eps}}
%
\title[LeNet]{Aprendizaje de máquina - LeNet5}
\author{Bruno, César}
\institute[ITAM]
%\date{November 7, 2019}
% \today will show current date. 
% Alternatively, you can specify a date.
%


\begin{document}
%
\begin{frame}
\titlepage
\end{frame}

%\begin{frame}
%\frametitle{Índice}
% \tableofcontents%[sections]
%\end{frame}

\begin{frame}
\section{Redes convolucionales}
\frametitle{LeNet-5}

\begin{itemize}
	\item LeNet-5 es una arquitectura de red convolucional desarrollada por Yann  LeCun, de las primeras en emplearse para el reconocimiento de caracteres escritos a mano.
	
	\item Consta de 7 capas, determinadas por parámetros de entrenamiento.
	
	\item Se alimenta de imágenes $32 \times 32$ pixeles.
	
	\item En total cuenta con 410,000 conexiones pero tiene 90,000 parámetros libres.
\end{itemize}





\end{frame}

\begin{frame}
\section{Arquitectura}
\frametitle{Arquitectura LeNet-5}

	\begin{figure}
		\includegraphics[scale=0.27]{images/lenet5_arch.png}
		\caption{Arquitectura de la Red Convolucional LeNet-5}
	 \end{figure}
\end{frame}

\begin{frame}
\frametitle{Arquitectura LeNet-5}
\begin{itemize}

	\item \textbf{input} Es una imagen de 32 X 32 en escala de grises
	
	\item \textbf{1a capa:} Capa convolucional que consta de 6 filtros de tamaño 5 X 5 y un stride de 1.
	La dimensión cambia a 28 X 28 X 6
	
	\item \textbf{2a capa:} capa de \emph{average pooling} de tamaño 2 X 2 y un stride de 2 (sub-sampling)\footnote{Reduce dimensiones de c/feature map retienendo información más importante}
	La dimensión baja a 14 X 14 X 6
	
	\item \textbf{3a capa:} capa convolucional que consta de 16 filtros de tamaño 5 X 5 y un stride de 1. En esta capa solo 10 de los 16 filtros están conectados
	La dimensión cambia a 10 X 10 X 16
	

\end{itemize}

\end{frame}

\begin{frame}
\frametitle{Arquitectura LeNet-5}
\begin{itemize}
	
	\item \textbf{4a capa:} nuevamente  capa de \emph{average pooling} de tamaño 2 X 2 y de 2 strides.
	La dimensión cambia a 5 X 5 X 16

	\item \textbf{5a capa:} es una capa convolucional que conecta la salida de la cuarta capa (400 parámetros) a una capa completamente conectada de 120 nodos de de tamaño 5 X 5
	La dimensión resultante es de 120
	
	\item \textbf{6a capa:} una capa igualmente completamente conectada que consta de 84 nodos, de la que se derivan de las salidas de los 120 nodos de la quinta capa.
	La dimensión reslutante es de 84.
	
\end{itemize}

\end{frame}

\begin{frame}
\frametitle{Arquitectura LeNet-5}
\begin{itemize}

	\item \textbf{7a capa:} consiste en clasificar la salida de la última capa en 10 clases relacionadas con los 10 dígitos que se entrenó principalmente para clasificar.

	\item La función de activación que se usa es una tangente hiperbólica.
	
\end{itemize}
\end{frame}

\begin{frame}
%\frametitle{Código de PyTorch LeNet-5}


	\begin{figure}
	\includegraphics[scale=0.5]{images/lenet.png}
	%\caption{Código de PyTorch LeNet-5}
\end{figure}

\end{frame}

\begin{frame}
\frametitle{Referencias}

\begin{itemize}
	\item \textbf{Gradient-Based Learning applied to document recognition}.  Y. Lecun; L. Bottou; Y. Bengio; P. Haffner See http://yann.lecun.com/exdb/publis/pdf/lecun-01a.pdf
	
	\item \textbf{Página electrónica de Y. Lecun dedicada a la red LeNet} http://yann.lecun.com/exdb/lenet/
\end{itemize}

\end{frame}


%
%\begin{frame}[fragile] % Notice the [fragile] option %
%\frametitle{Verbatim}
%\begin{example}[Putting Verbatim]
%\begin{verbatim}
%\begin{frame}
%\frametitle{Outline}
%\begin{block}
%{Why Beamer?}
%Does anybody need an introduction to Beamer?
%I don't think so.
%\end{block}
% Extra carriage return causes problem with verbatim %
%\end{frame}\end{verbatim} 
%\end{example}
%\end{frame}
 
%\begin{frame}[fragile]  % notice the fragile option, since the body
			% contains a verbatim command
%Example of the \verb|\cite| command to give a reference is below:
%Example of citation using \cite{key1} follows on.
%\end{frame}
 
% \begin{frame}
% \section{Bibliografía}
% \frametitle{Referencias}
% \footnotesize{
% \begin{thebibliography}{99}
%  \bibitem[Morita, 2010]{key1} J. Nagata, K. Morita (1989)
%  \newblock Topics On General Topology.
%  \newblock \emph{Elsevier Science Publisher B.V.} 15(6), 203 -- 243.

% \bibitem[VanMill, 2010]{key1} J. Van Mill ; M. Husek (1992)
%  \newblock Recent Progress In General Topology.
%  \newblock \emph{Elsevier Publications} p. 375.

% \bibitem[MacLane, 2010]{key1} J. S. Mac Lane(1971)
%  \newblock Categories for the working mathematician,.
%  \newblock \emph{Springer} p. 375.

%  \bibitem[Ishii, 2010]{key1} Tadashi Ishii (1969)
%  \newblock On Tychonoff Functor and $w$-Compactness.
%  \newblock \emph{Topology Appl.} 11, 175 -- 187.

%  \bibitem[Ishii, 2010]{key1} T. Hoshina; K. Morita (1980)
%  \newblock On Regular Products Of Topological Spaces.
%  \newblock \emph{Topology Appl.} 11, 47 -- 57.
% \end{thebibliography}
% }
% \end{frame}


 
% \begin{frame}
% %\section{Bibliografía}
% \frametitle{Referencias}
% \footnotesize{
% \begin{thebibliography}{99}
%  \bibitem[Porter, 2010]{key1}J. R. Porter ; R. Grant Woods  (1987)
%  \newblock Extensions and Absolutes of Hausdorff Spaces.
%  \newblock \emph{Springer-Verlag} 856.


%  \bibitem[Simon, 2010]{key1} Petr Simon (1984)
%  \newblock Completely regular modification and products.
%  \newblock \emph{Commentationes Mathematicae Universitatis Carolinae} 25(1), 121--128.



%  \bibitem[Puppier, 2010]{key1} René Puppier (1969)
%  \newblock La Completion Universelle D'un Produit D'espaces Completement Reguliers .
%  \newblock \emph{Publ. Dept. Math, Lyon} 254, 342--351.



% \end{thebibliography}
% }
% \end{frame}
% 
% End of slides
\end{document} 
